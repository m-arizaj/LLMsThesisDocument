% !TEX root = ../thesis-example.tex
%
\chapter{Conclusiones}
\label{chapter5}

Durante el desarrollo del proyecto se lograron la mayoría de los objetivos propuestos, y se construyó un generador automático de pruebas funcional. A continuación se presentan las conclusiones del proceso y el trabajo futuro.

\section{Objetivos cumplidos}

Gracias a la implementación de esta herramienta se obtuvo un producto inicial que genera pruebas basado en los modelos producidos por RIP. Durante el proceso se adquirieron conocimientos sobre exploración de aplicaciones móviles y generación de código java. Además, se logró el objetivo principal, integrar multi-modelos en la generación de pruebas. Se tuvieron en cuenta los 3 modelos, contexto, dominio y GUI, lo cual es el factor que diferencia la solución de otras ya existentes.


\section{Limitaciones del desarrollo}
\begin{itemize}
	\item Como se mencionó anteriormente, la generación de pruebas está enfocada exclusivamente en aplicaciones Android. Esto porque es el sistema operativo donde se presentan mayores problemas de desfragmentación y ofrece la facilidad de explorar las aplicaciones mediante herramientas como ADB.
	
	\item Debido al alcance del proyecto, el desarrollador debe tener el código fuente de la aplicación para poder ejecutar la prueba
	
	\item La extracción del modelo de dominio está basada en los componentes básicos de Android, por lo tanto, los que sean personalizados o de otras librerías no serán reconocidos en dicho modelo.
	
	\item El desarrollador debe tener derechos de superusuario sobre el dispositivo o correr las pruebas en un emulador para realizar algunas acciones.
	
	
\end{itemize}


\section{Trabajo futuro}

El proyecto presenta varias oportunidades de mejora como lo son:

\begin{itemize}
	\item Ejecución de la prueba sin tener acceso al código fuente.
	\item Extender los componentes que son detectados en los diferentes modelos.
	\item Tener una interfaz para el uso de la herramienta.
	\item Permitir al usuario ingresar valores personalizados para campos que lo requieran como los de usuario y contraseña.
	
	
\end{itemize}

