\subsection{General Principles}

\subsubsection*{Introduction}
The \textbf{General Principles} metric in DevEval evaluates the overall quality of software design artifacts produced by LLMs. Rather than assessing correctness with respect to a specification (handled separately by the Faithfulness metric), General Principles focuses on the \textbf{internal design quality} of UML diagrams and architecture designs.

The metric aims to capture desirable software engineering characteristics such as cohesion, decoupling, uniformity, integration, practicability, interaction complexity, and alignment with widely accepted design conventions. According to the paper, ``The general principles metric plays a crucial role, with each task sharing common elements while maintaining specific criteria.''

\subsubsection*{Formula}
DevEval does not define a mathematical formula for the General Principles metric. Instead, it is a \textbf{qualitative, LLM-as-a-judge evaluation} that relies on rubric-style criteria. Each principle is assessed by a judge LLM based on specific rubric text provided in the benchmark's appendix.

\subsubsection*{Variants (By Design Subtask)}
While there are no distinct named variants, the metric adapts its criteria depending on the specific design artifact being evaluated:

\begin{enumerate}
    \item \textbf{UML Class Diagrams}:
    \begin{itemize}
        \item \textit{Cohesion and Decoupling}: ``The design should aim for high cohesion within individual classes and low coupling between different classes.''
        \item \textit{Complexity}: Evaluated using metrics such as total number of classes, average methods per class, and inheritance depth.
        \item \textit{Practicability}: The design must be readable, understandable, and evidently modular.
    \end{itemize}

    \item \textbf{UML Sequence Diagrams}:
    \begin{itemize}
        \item \textit{Uniformity and Integration}: The design should demonstrate a consistent style and integrated approach.
        \item \textit{Interaction Complexity}: Focuses on the number of messages, depth of nested calls, and participating objects.
        \item \textit{Practicability}: Focuses on readability, understandability, and clarity in interactions.
    \end{itemize}

    \item \textbf{Architecture Design}:
    \begin{itemize}
        \item \textit{Uniformity and Integration}: Ensuring seamless component interaction with high cohesion and decoupling.
        \item \textit{Distinction Between Design and Coding}: Recognizing the design process as distinct from implementation.
        \item \textit{Conformance}: Evaluating adherence to community and industry standards.
    \end{itemize}
\end{enumerate}

\subsubsection*{Applications in Software Engineering}
General Principles maps directly onto classical software engineering design quality attributes, making it essential for evaluating whether models can generate designs that real developers could meaningfully implement. Key attributes include:
\begin{itemize}
    \item \textbf{Cohesion \& Decoupling}: Improving maintainability and testing.
    \item \textbf{Practicability}: Mirroring human judgments of usability and clarity.
    \item \textbf{Uniformity}: Ensuring systematic rather than piecemeal design.
    \item \textbf{Interaction Complexity}: Crucial for distributed and concurrent systems.
\end{itemize}

\subsubsection*{Interpretation}
\begin{itemize}
    \item \textbf{High Score}: Indicates high cohesion, low coupling, readable structure, appropriate complexity, and adherence to community standards.
    \item \textbf{Low Score}: Suggests disorganized structure, excessive coupling, poor readability, or inconsistent architecture layouts.
\end{itemize}

Empirically, results show strong variability: GPT-4-Turbo achieves a \textbf{97.9\% win rate} against GPT-3.5-Turbo, while smaller models score significantly lower, underscoring that design quality requires strong reasoning capabilities.

% añadir esta entrada al archivo .bib
% @article{li2024prompting,
%   title={Prompting large language models to tackle the full software development lifecycle: A case study},
%   author={Li, B. and Wu, W. and Tang, Z. and Shi, S.},
%   journal={arXiv preprint arXiv:2403.08604},
%   year={2024},
%   doi={10.48550/arXiv.2403.08604}
% }