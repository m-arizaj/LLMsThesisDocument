\subsection{Helpfulness}

\subsubsection*{Introduction}
Helpfulness is a key category of \textbf{human-centered evaluation} for Large Language Models (LLMs). It is a qualitative assessment that measures how well an LLM's response satisfies a user's request, answers their question, or accomplishes the instructed task.

It is the primary metric for evaluating a model's utility and capability, often balanced against ``Harmlessness'' to ensure that responses are not only safe but also functional and useful.

\subsubsection*{Definition}
Helpfulness is defined as the measure of a model's ability to provide high-quality, relevant, and accurate answers that align with the user's intent. It quantifies the practical utility of the system.

\subsubsection*{Measurement Methodology}
Helpfulness is measured primarily through \textbf{Human Evaluation}, often using a ``battle'' or pairwise comparison format:

\begin{itemize}
    \item \textbf{Chatbot Arena}: Human users interact with two different anonymous models, ask them the same question, and vote for which model provided the ``better'' (i.e., more helpful and thorough) response.
    \item \textbf{HELM}: Uses human evaluations to score model outputs on various criteria, including accuracy and helpfulness for specific tasks.
\end{itemize}

In these evaluations, raters score based on relevance, accuracy, completeness, and overall quality.

\subsubsection*{Purpose}
The purpose of measuring helpfulness is to \textbf{quantify the utility and performance} of an LLM from a user's perspective. It serves as the main metric for determining if a model is proficient at following instructions, which is the primary goal of assistant-style models.

\subsubsection*{Domains}
\begin{itemize}
    \item Human-centered Evaluation
    \item LLM Evaluation / NLP
    \item Chatbot Performance
\end{itemize}

\subsubsection*{Benchmarks}
\begin{itemize}
    \item \textbf{HELM} (Holistic Evaluation of Language Models)
    \item \textbf{Chatbot Arena}
    \item \textbf{MT-Bench} (often used in conjunction with Chatbot Arena)
\end{itemize}

\subsubsection*{Advantages}
\begin{itemize}
    \item \textbf{Aligns with User Goals}: Directly measures what most users care about—getting a good answer.
    \item \textbf{Captures Nuance}: Human evaluators can reward responses that are not just factually correct but also well-written, comprehensive, and appropriately toned.
    \item \textbf{Gold Standard for Utility}: It serves as the most reliable ``ground truth'' for a model's practical capability, often revealing gaps that automated metrics miss.
\end{itemize}

\subsubsection*{Limitations}
\begin{itemize}
    \item \textbf{Subjectivity}: What one user finds ``helpful,'' another might find overly verbose or simplistic. The metric is inherently subjective.
    \item \textbf{Scalability}: Collecting human judgments is expensive and slow compared to running automated computational metrics.
    \item \textbf{Evaluator Bias}: Ratings can be influenced by the knowledge, preferences, and potential biases of the human evaluators.
\end{itemize}

\subsubsection{Additional References}

This metric is referenced and/or used in the following paper(s):


\sloppy
\cite{
Chang2023SurveyLLMs,
Liang2022HELM,
zheng2023judging,
}
\fussy
