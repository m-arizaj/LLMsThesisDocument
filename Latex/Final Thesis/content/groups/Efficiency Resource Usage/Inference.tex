\subsection{Inference (Efficiency and Verification)}

\subsubsection*{Introduction}
In the evaluation of Large Language Models (LLMs) and the systems they generate, ``Inference'' describes metrics in two distinct categories:

\begin{itemize}
    \item \textbf{Inference Efficiency}: The performance of a model during the generation phase (speed, resource consumption, runtime).
    \item \textbf{Behavioral Model Inference}: A specialized verification metric in Software Engineering referring to the success rate of automatically learning a correct behavioral model (e.g., an automaton) to validate functional correctness.
\end{itemize}

\subsubsection*{1. Inference Efficiency (Runtime / Speed)}

\textbf{Definition} \\
Inference Efficiency metrics measure the computational cost and speed of a language model when performing its primary function: generating text. HELM (Holistic Evaluation of Language Models) proposes several specific metrics:
\begin{itemize}
    \item \textbf{Raw Runtime}: Actual wall-clock time including network noise.
    \item \textbf{Denoised Inference Runtime}: Estimated runtime with contention (queuing) noise factored out.
    \item \textbf{Idealized Inference Runtime}: A standardized metric estimating runtime on a uniform, optimized hardware stack (e.g., NVIDIA A100) for fair comparison.
\end{itemize}

\textbf{Formula (HELM Idealized Runtime)} \\
The total time for an idealized runtime is modeled as a function of prompt processing time and token generation time:

\begin{equation}
    \text{Total Time} = F(N_{\text{prompt}}) + g \cdot N_{\text{output}}
\end{equation}

\noindent where:
\begin{itemize}
    \item $F$: Function modeling the runtime of processing the prompt tokens ($N_{\text{prompt}}$).
    \item $g$: Runtime (cost) of generating each additional output token.
    \item $N_{\text{output}}$: Number of output tokens generated.
\end{itemize}

\textbf{Purpose} \\
To measure the practical performance of an LLM. High efficiency is critical for user-facing applications (chatbots) and autonomous agents requiring fast responses.

\textbf{Limitations} \\
Comparing efficiency is difficult due to non-model factors like hardware types (GPUs/TPUs), software optimizations, and API server contention.

\subsubsection*{2. Behavioral Model Inference (Success Rate)}

\textbf{Definition} \\
Specific to Learning-Based Testing and Language-Driven Engineering (LDE), this metric does \textbf{not} measure computational speed. Instead, it measures the \textbf{success rate (\%)} of automatically \textit{inferring} a correct behavioral model (e.g., a Mealy machine) from a generated application.

For example, in system migration (e.g., JS to TS), a tool tests the resulting application to infer its behavior. The success rate reflects whether this inferred model is correct, validating the migration.

\textbf{Purpose} \\
To automatically validate functional correctness and reliability. It verifies that the generated system behaves as intended or identifies behavioral differences using ``difference automata.''

\subsubsection*{Comparative Summary}
\begin{center}
\begin{tabular}{|p{3cm}|p{2.5cm}|p{5cm}|p{2.5cm}|}
\hline
\textbf{Metric Category} & \textbf{Measures} & \textbf{Description} & \textbf{Typical Domain} \\
\hline
Inference Efficiency & Efficiency & Computational cost, time (s), or tokens/sec during text generation. & LLM Evaluation, Agents \\
\hline
Behavioral Model Inference & Verification & Success rate (\%) of automatically deriving a correct behavioral automaton from a system. & Software Engineering, LDE \\
\hline
\end{tabular}
\end{center}

% añadir estas entradas al archivo .bib

% @article{busch2025llm,
%   title={LLM-based code generation and system migration in language-driven engineering},
%   author={Busch, D. and Bainczyk, A. and Smyth, S.},
%   journal={International Journal on Software Tools for Technology Transfer},
%   volume={27},
%   pages={137--147},
%   year={2025},
%   doi={10.1007/s10009-025-00798-x}
% }

% @article{liang2022holistic,
%   title={Holistic evaluation of language models (HELM)},
%   author={Liang, P. and Bommasani, R. and Lee, T. and others},
%   journal={arXiv preprint arXiv:2211.09110},
%   year={2022},
%   doi={10.48550/arXiv.2211.09110}
% }

% @article{wang2023survey,
%   title={A survey on large language model based autonomous agents},
%   author={Wang, L. and Ma, C. and Feng, X. and others},
%   journal={arXiv preprint arXiv:2308.11432},
%   year={2023},
%   doi={10.48550/arXiv.2308.11432}
% }