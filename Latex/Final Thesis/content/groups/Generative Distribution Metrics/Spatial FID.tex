\subsection{Spatial FID (sFID)}

\subsubsection*{Definition}
\textbf{Spatial FID (sFID)} is an alternative metric for ranking generative models, introduced as a variation of the standard Fréchet Inception Distance (FID).

The key difference lies in the source of the feature representations. While the standard FID uses activations from the final \textit{pool3} layer (2048-dimensional) of the Inception-V3 network, \textbf{sFID computes the FID using activations from an earlier, intermediate layer}. Specifically, it uses the \textbf{mixed\_6/conv layer} of the Inception-V3 network.

\subsubsection*{Formula}
The underlying mathematical calculation for sFID is the same Fréchet Distance (FD) formula used by the standard FID, but applied to different feature vectors:

\begin{equation}
    \text{sFID} = ||\mu_r - \mu_g||_2^2 + \text{Tr}(\Sigma_r + \Sigma_g - 2(\Sigma_r \Sigma_g)^{1/2})
\end{equation}

\noindent where:
\begin{itemize}
    \item $\mu_r, \mu_g$: The sample means of the real and generated representations derived from the \textbf{mixed\_6/conv} layer.
    \item $\Sigma_r, \Sigma_g$: The sample covariances of the real and generated representations derived from the same layer.
\end{itemize}

\subsubsection*{Purpose}
The purpose of sFID is to serve as an alternative to the standard FID for ranking generative models. By using an intermediate layer, it presumably captures different features—specifically, more spatial and less class-abstracted information—than the final pooling layer used by standard FID.

\subsubsection*{Domains}
\begin{itemize}
    \item Generative Models
    \item Image Generation
\end{itemize}

\subsubsection*{Limitations}
\begin{itemize}
    \item \textbf{Inception-V3 Dependency}: The sFID metric, by definition, relies on the Inception-V3 network architecture.
    \item \textbf{Non-Transferable}: Because it is tied to a specific internal layer (mixed\_6/conv) of Inception-V3, the metric is not typically reported for or compatible with other network architectures.
\end{itemize}

% añadir esta entrada al archivo .bib
% @article{nash2021generating,
%   title={Generating images with sparse representations},
%   author={Nash, C. and Menick, J. and Dieleman, S. and Battaglia, P. W.},
%   journal={arXiv preprint arXiv:2103.03841},
%   year={2021},
%   doi={10.48550/arXiv.2103.03841}
% }