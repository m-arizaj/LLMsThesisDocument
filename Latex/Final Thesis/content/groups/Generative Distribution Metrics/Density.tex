\subsection{Density}

\subsubsection*{Introduction}
Density is an automatic evaluation metric used to diagnose the performance of generative models, particularly in image synthesis. It was introduced by Naeem et al. (2020) as part of a metric pair, \textit{Density and Coverage}, designed to reliably and separately measure a model's fidelity and diversity, respectively.

These metrics work by analyzing the $k$-nearest neighbors ($k$-NN) relationships between the distributions of real and generated samples in a feature space (produced by an encoder). Density is specifically used as a proxy for the \textit{fidelity} of the generated samples—that is, how ``realistic'' they are.

\subsubsection*{Definition}
Density quantifies how well-populated the generated samples are relative to the real samples in the feature space. Specifically, it counts how many real-sample neighborhood spheres contain the generated sample. A higher Density value suggests better fidelity, indicating that the generated samples fall into regions that are densely populated by real training samples.

The metric is defined mathematically as:

\begin{equation}
    \text{Density}(\{x_i^g\}_{i=1}^n, \{x_j^r\}_{j=1}^m) = \frac{1}{kn} \sum_{i=1}^n \sum_{j=1}^m \mathbb{1}\left(x_i^g \in B(x_j^r, \text{NND}_k(x_j^r))\right)
\end{equation}

\noindent where:
\begin{itemize}
    \item $\{x^g\}$: the set of $n$ generated samples.
    \item $\{x^r\}$: the set of $m$ real (training) samples.
    \item $\text{NND}_k(x_j^r)$: the distance from a real sample $x_j^r$ to its $k$-th nearest neighbor within the real set.
    \item $B(x,r)$: a Euclidean ball centered at $x$ with radius $r$.
    \item $k$: the number of neighbors (the original paper uses $k=5$).
\end{itemize}

\subsubsection*{Purpose}
Density is used as a proxy to measure sample \textbf{fidelity} (quality or realism).

\subsubsection*{Applications}
Its primary application is diagnosing the performance of generative models. The original paper uses it to evaluate image generation models on datasets like CIFAR10, ImageNet, FFHQ, and LSUN-Bedroom.

\subsubsection*{Limitations}
\begin{itemize}
    \item Its calculation is based on $k$-nearest neighbors and can be sensitive to the choice of $k$ and the number of samples used (e.g., $k=5$ and 10,000 samples).
    \item Research suggests that other metrics, like Fréchet Distance (FD) with a robust encoder (e.g., DINOv2), may correlate better with human judgments of fidelity than $k$-NN metrics like Precision (and by extension, Density).
\end{itemize}

\subsubsection{Additional References}

This metric is referenced and/or used in the following papers:


\sloppy
\cite{
Stein2023MetricFlaws,
Naeem2020Reliable,
}
\fussy