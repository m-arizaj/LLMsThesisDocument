\subsection{SIDE}

\subsubsection*{Definition}
\textbf{SIDE} is an \textbf{embedding-based metric} specifically designed to evaluate the quality of code summaries. It differentiates itself from other embedding-based metrics like BERTScore by applying \textbf{contrastive learning}.

Instead of just measuring direct similarity, SIDE enhances the evaluation by learning to pull similar (positive) code-summary pairs closer together in the embedding space while pushing dissimilar (negative) pairs apart. This approach allows it to better capture the semantic adequacy of a generated summary.

\subsubsection*{Formula (General Idea)}
SIDE works by encoding both the generated artifact (the code summary) and a reference summary into high-dimensional vectors (embeddings) and then measuring the similarity between them.

While the core calculation relies on similarity, the embeddings $\mathbf{e}$ are optimized via contrastive learning. The metric can be conceptually represented as:

\begin{equation}
    \text{SIDE}(S_{gen}, S_{ref}) \sim \cos(\mathbf{e}_{gen}, \mathbf{e}_{ref})
\end{equation}

\noindent where:
\begin{itemize}
    \item $\mathbf{e}_{gen}$ and $\mathbf{e}_{ref}$ are the vector representations of the generated and reference summaries, respectively.
    \item The embedding space is trained to minimize the distance between semantic equivalents and maximize the distance between dissimilar pairs.
\end{itemize}

\subsubsection*{Purpose}
The primary purpose of SIDE is to assess the quality of machine-generated code summaries. It is used as an automatic metric to compare a candidate summary against a human-written reference summary, aiming for a closer correlation with human judgment than traditional n-gram metrics.

\subsubsection*{Domains}
\begin{itemize}
    \item Software Engineering
    \item Code Summarization
\end{itemize}

\subsubsection*{Advantages}
\begin{itemize}
    \item \textbf{State-of-the-Art}: It is considered a state-of-the-art metric specifically designed for the code summarization task.
    \item \textbf{Task-Specific}: Unlike general text-based metrics (like BLEU or ROUGE), SIDE's use of contrastive learning is tailored specifically for the nuances of code documentation.
\end{itemize}

\subsubsection*{Limitations}
\begin{itemize}
    \item \textbf{Similarity vs. Correctness}: Like other embedding-based metrics, SIDE ``approximates correctness through similarity.'' This is a limitation because ``similarity does not always align with correctness.'' A summary could be syntactically different but semantically correct, or vice versa, leading to potential scoring discrepancies.
\end{itemize}

% añadir estas entradas al archivo .bib

% @inproceedings{mastropaolo2024evaluating,
%   title={Evaluating code summarization techniques: A new metric and an empirical characterization},
%   author={Mastropaolo, A. and Ciniselli, M. and Di Penta, M. and Bavota, G.},
%   booktitle={Proceedings of the 46th IEEE/ACM International Conference on Software Engineering},
%   pages={1--13},
%   year={2024},
%   publisher={ACM},
%   doi={10.1145/3597503.3639174}
% }

% @article{zhou2025sejury,
%   title={SE-Jury: An LLM-as-ensemble-judge metric for narrowing the gap with human evaluation in SE},
%   author={Zhou, X. and Kim, K. and Zhang, T. and Lo, D. and Han, D. G.},
%   journal={arXiv preprint arXiv:2505.20854},
%   year={2025},
%   doi={10.48550/arXiv.2505.20854}
% }