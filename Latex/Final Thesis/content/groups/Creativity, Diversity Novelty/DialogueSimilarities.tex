\subsection{Dialogue Similarities}

\subsubsection*{Introduction}
Dialogue Similarities is a type of \textit{Human Similarity metric} used in the objective evaluation of LLM-based autonomous agents. This metric is designed to quantify the degree to which an agent's behavior, specifically its conversational output, ``closely resembles that of humans.''

It is an objective, quantitative metric often grouped with other human similarity metrics—such as \textit{trajectory/location accuracy} and \textit{mimicry of human responses}—to assess an agent's performance in simulating human behavior.

\subsubsection*{Definition}
Dialogue Similarities is formally defined within the category of Human Similarity metrics. It serves as a quantitative measure that evaluates:

\begin{quote}
``The degree to which the agent behaviors closely resembles that of humans.''
\end{quote}

It is cited specifically as a ``typical example'' of such metrics, distinguishing itself by focusing on the semantic and stylistic alignment of the agent's dialogue with human baselines rather than just functional correctness.

\subsubsection*{Purpose}
The primary purpose of measuring dialogue similarities is to assess the agent's \textbf{human simulation performance}. A high score in dialogue similarity indicates that the agent is effective at mimicking human-like conversational behavior, which is crucial for agents designed for social interaction or role-playing.

\subsubsection*{Applications}
This metric is primarily applied in:
\begin{itemize}
    \item The \textbf{objective evaluation} of autonomous agents.
    \item Protocols involving \textbf{Social evaluation} or ``human simulation,'' where the goal is indistinguishability from human actors.
\end{itemize}

\subsubsection*{Limitations}
\begin{itemize}
    \item As a form of objective evaluation, Dialogue Similarities provides quantitative insights but may not ``perfectly measure all types of agent capabilities.''
    \item It focuses on resemblance rather than utility or safety; therefore, such objective metrics are considered essential complements to, but not replacements for, subjective assessment.
\end{itemize}


\subsubsection{Additional References}

This metric is referenced and/or used in the following paper(s):


\sloppy
\cite{
wang2024survey,
Zhou2023CodeBERTScore,
}
\fussy