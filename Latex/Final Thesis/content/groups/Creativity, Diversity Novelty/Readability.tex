\subsection{Readability}

\textbf{Readability} is a criterion used to evaluate the \textbf{Linguistic Quality} of text generated by Large Language Models (LLMs).

In the context of recent surveys (e.g., Chang et al., 2024), Readability is not treated as a standalone automatic metric but as a key component of \textbf{Fluency}, which is a fundamental criterion for the \textbf{Human Evaluation} of LLMs. It assesses the model's ability to produce content that flows smoothly, is grammatically correct, maintains a consistent tone, and ensures a seamless user experience.

In this specific context, Readability is treated as a component of Fluency within \textbf{Human Evaluation} rather than a specific automatic formula.

It is a \textbf{qualitative assessment} made by human evaluators who judge the quality of the generated text. This contrasts with automatic metrics like ROUGE or F1-Score which are calculated based on token overlap.

\subsubsection*{Purpose}
The primary purpose is to assess the quality of the generated text from a human perspective. This evaluation aims to:
\begin{itemize}
    \item Ensure the text ``flows smoothly.''
    \item Verify that the text is ``grammatically correct.''
    \item Confirm the model maintains a ``consistent tone and style.''
    \item Measure if the text provides a ``seamless user experience.''
    \item Determine if the model avoids awkward expressions and abrupt shifts.
\end{itemize}

\subsubsection{Applications}

In the context of software engineering for Large Language Models (LLMs), Readability is applied primarily during the \textbf{System Validation} and \textbf{Feature Engineering} phases to ensure outputs meet user requirements.

\begin{itemize}
    \item \textbf{Validation of Technical Translation Systems}:
    In safety-critical domains like healthcare software, readability is a functional requirement. It is applied to validate systems designed to translate complex technical jargon into plain language. For instance, engineers evaluate whether models effectively translate radiology reports into "easily understandable language" for patients, ensuring the system bridges the gap between expert data and layperson comprehension \cite{Chang2023SurveyLLMs}.

    \item \textbf{Trade-off Analysis in Summarization Algorithms}:
    In the engineering of summarization tools, readability is used to diagnose system behavior under constraints. Van der Lee et al. (2019) note that in summarization tasks involving high-volume data, "readability and adequacy are mutually conflicting goals." Engineers use this metric to calibrate the model, finding the optimal balance where the text remains legible without losing critical information \cite{vanderlee2019best}.

    \item \textbf{Regression Modeling for Quality Estimation}:
    To scale evaluation without constant human intervention, software engineers use readability principles to build automated testing tools. Objective metrics (such as average word length or parse tree height) are used as features in regression models to obtain a "text quality score" that approximates human judgment, allowing for automated monitoring of system performance \cite{vanderlee2019best}.

    \item \textbf{User Experience (UX) Testing for EdTech}:
    In educational software, readability is evaluated to ensure AI-generated feedback is not only correct but pedagogically effective. It serves as a metric to verify that the system maintains a consistent tone and provides a "seamless user experience," which is crucial for maintaining student engagement in automated tutoring systems \cite{Chang2023SurveyLLMs}.
\end{itemize}

\subsubsection*{Advantages}
\begin{itemize}
    \item \textbf{More Accurate Feedback}: As a human evaluation metric, it is ``closer to the actual application scenario'' and provides feedback that is more comprehensive than automatic metrics.
    \item \textbf{Reliable for Open Generation}: Considered ``more reliable'' for open-ended tasks where reference-based metrics (like BLEU) are often unsuitable.
    \item \textbf{Captures Nuance}: Assesses subtle linguistic qualities like flow and tone that automatic formulas fail to capture.
\end{itemize}

\subsubsection*{Limitations}
\begin{itemize}
    \item \textbf{High Variance}: As a human-judged metric, it can suffer from ``high variance and instability.''
    \item \textbf{Subjectivity}: Evaluation can be heavily influenced by ``cultural and individual differences'' among evaluators.
    \item \textbf{Requires Rigor}: Reliability depends on factors like the number of evaluators, their expertise, and the clarity of evaluation rubrics.
\end{itemize}

\subsubsection{Additional References}

This metric is referenced and/or used in the following paper(s):


\sloppy
\cite{
Chang2023SurveyLLMs,
vanderlee2019best,
}
\fussy

