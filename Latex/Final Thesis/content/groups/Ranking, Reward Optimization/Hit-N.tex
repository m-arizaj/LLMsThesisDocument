\subsection{Hit-N}

\subsubsection*{Definition}
\textbf{Hit-N} (also referred to as Hit-N (multi-defects)) is an evaluation metric used in fault localization. It is specifically ``designed for multi-defects.''

The metric measures the number of bugs (out of a set of known bugs) for which a fault localization tool's predicted set of suspicious code elements (e.g., the top-ranked statements) successfully ``contains at least N faulty statements.''

\subsubsection*{Formula (General Idea)}
The metric is defined qualitatively based on a counting mechanism. It can be expressed as:

\begin{equation}
    \text{Hit-N} = \sum_{b \in B} \mathbb{1}(|S_{\text{predicted}}(b) \cap S_{\text{faulty}}(b)| \geq N)
\end{equation}

\noindent where:
\begin{itemize}
    \item $B$: The set of bugs in the multi-bug benchmark.
    \item $S_{\text{predicted}}(b)$: The set of suspicious statements predicted by the tool for bug $b$.
    \item $S_{\text{faulty}}(b)$: The set of actual faulty statements for bug $b$.
    \item $N$: The threshold number of faulty statements required for a ``hit.''
\end{itemize}

\subsubsection*{Purpose}
The primary purpose of Hit-N is to evaluate the effectiveness of fault localization techniques in scenarios involving multiple defects. Unlike metrics that might only measure the rank of the \textit{first} fault found (like Mean First Rank - MFR), Hit-N assesses a tool's ability to identify faulty statements across multiple, potentially interacting, bugs within the same project.

\subsubsection*{Domains}
\begin{itemize}
    \item Fault Localization Evaluation
    \item LLM Evaluation (as applied to code/debugging tasks)
\end{itemize}

\subsubsection*{Advantages}
\begin{itemize}
    \item \textbf{Designed for Multi-Defect Scenarios}: Its main advantage is its specific design for evaluating performance against multiple bugs, which is often a more realistic scenario than the single-bug assumption.
    \item \textbf{Measures Set-Based Success}: It evaluates the \textit{set} of predicted locations rather than just the single highest-ranked location.
\end{itemize}

\subsubsection*{Limitations}
\begin{itemize}
    \item \textbf{Infrequent Use}: Survey literature indicates that this metric is not commonly used (listing only one use case in reviewed literature).
    \item \textbf{Niche Application}: Its utility is limited to multi-defect fault localization; it is not a general-purpose metric. Most existing fault localization research still assumes only one bug is present.
\end{itemize}

% añadir esta entrada a tu archivo .bib
% @inproceedings{li2022fault,
%   title={Fault localization to detect co-change fixing locations},
%   author={Li, Y. and Wang, S. and Nguyen, T. N.},
%   booktitle={Proceedings of the 30th ACM Joint European Software Engineering Conference and Symposium on the Foundations of Software Engineering},
%   pages={659--671},
%   year={2022},
%   publisher={ACM},
%   address={New York},
%   doi={10.1145/3540250.3549137}
% }