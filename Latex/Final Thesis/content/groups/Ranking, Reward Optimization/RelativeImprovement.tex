\subsection{Relative Improvement (RImp)}

\subsubsection*{Definition}
\textbf{Relative Improvement (RImp)} is an evaluation metric used in Fault Localization (FL) to assess the effectiveness of a specific FL technique.

It is calculated by comparing the total number of statements a developer would need to examine to find all faults \textit{using} a specific FL approach versus the total number of statements they would need to examine \textit{without} using that approach (the baseline).

\subsubsection*{Formula}
Although often described qualitatively as a comparison of effort, the standard calculation for relative improvement can be expressed as:

\begin{equation}
    \text{RImp} = \frac{\text{Statements}_{\text{baseline}} - \text{Statements}_{\text{FL\_technique}}}{\text{Statements}_{\text{baseline}}} \times 100\%
\end{equation}

\noindent where:
\begin{itemize}
    \item $\text{Statements}_{\text{baseline}}$: Total statements examined to find all faults without the FL approach.
    \item $\text{Statements}_{\text{FL\_technique}}$: Total statements examined to find all faults using the FL approach.
\end{itemize}

\subsubsection*{Purpose}
The purpose of RImp is to quantify the \textbf{practical benefit or efficiency gain} provided by a fault localization technique. It directly measures the reduction in developer effort (in terms of code statements to review) required to locate all faults, providing a clear indicator of a technique's value.

\subsubsection*{Domains}
\begin{itemize}
    \item Software Engineering
    \item Fault Localization (FL)
    \item Deep-Learning-based Fault Localization (DLFL)
\end{itemize}

\subsubsection*{Advantages}
\begin{itemize}
    \item \textbf{Measures Practical Effort}: Directly quantifies the reduction in developer effort, making it a practical metric for assessing a tool's real-world usefulness.
    \item \textbf{Clear Interpretation}: A higher RImp score clearly indicates a more effective localization technique compared to the baseline.
\end{itemize}

% añadir estas entradas al archivo .bib

% @article{chen2025deep,
%   title={Deep learning-based software engineering: Progress, challenges, and opportunities},
%   author={Chen, X. and Hu, X. and Huang, Y. and Ma, L. and Wang, H. and Wang, J. and others},
%   journal={Science China Information Sciences},
%   volume={68},
%   pages={111102},
%   year={2025},
%   doi={10.1007/s11432-023-4127-5}
% }

% @article{zhang2017deep,
%   title={Deep learning-based fault localization with contextual information},
%   author={Zhang, Z. and Lei, Y. and Tan, Q. and Mao, X. and Zeng, P.},
%   journal={IEICE Transactions on Information and Systems},
%   volume={E100.D},
%   number={12},
%   pages={3027--3030},
%   year={2017},
%   doi={10.1587/transinf.2017EDL8143}
% }

% @article{zhang2023context,
%   title={Context-aware neural fault localization},
%   author={Zhang, Z. and Lei, Y. and Mao, X. and Zeng, P.},
%   journal={IEEE Transactions on Software Engineering},
%   volume={49},
%   number={7},
%   pages={3862--3883},
%   year={2023},
%   doi={10.1109/TSE.2023.3279125}
% }