\subsection{Cohesion and Decoupling}

\subsubsection{Introduction}

Cohesion and Decoupling are foundational principles for assessing the structural quality of software designs. In DevEval, these principles are explicitly used to evaluate LLM-generated software design artifacts — including UML class diagrams, sequence diagrams, and architecture designs. The authors describe these principles as core criteria shared across all design subtasks:

\begin{quote}
“For all the subtasks, principles like cohesion and decoupling, and practicability are fundamental. Cohesion and decoupling emphasize the importance of clarity and functionality within individual elements (classes or sequences) and reducing dependencies between different components.” \cite{Li2024FullSDLC}
\end{quote}

Cohesion measures how well the responsibilities within a module belong together, while decoupling (low coupling) measures the independence of modules from one another. Together, they form a structural assessment of how well a model can produce modular, maintainable, and well-organized designs during the early phases of the software development lifecycle \cite{Li2024FullSDLC}.

\subsubsection{Formula}

DevEval does not introduce explicit numerical formulas for cohesion or coupling. Instead, these principles function as qualitative evaluation criteria used by an LLM-as-a-judge framework to compare design quality \cite{Li2024FullSDLC}. The paper clarifies that these principles guide structural reasoning. Because DevEval uses pairwise LLM-based judgments rather than static numerical computation, the metric is conceptual rather than algebraic.

However, cohesion and coupling can be understood using traditional software engineering theory:
\begin{itemize}
    \item \textit{Cohesion}: similarity or thematic unity of responsibilities within a class or module.
    \item \textit{Coupling}: degree of interdependence between classes or modules.
\end{itemize}

A conceptual normalized structural quality score could therefore be represented as:

\[
\text{DesignStructuralQuality} = f(\text{Cohesion}, \text{Coupling})
\]

Where:
\begin{itemize}
    \item Higher cohesion improves the score.
    \item Higher coupling reduces it.
\end{itemize}

Although DevEval does not formalize this as a numeric metric, it consistently applies these principles within its comparative design evaluation \cite{Li2024FullSDLC}.

\subsubsection{Variants}

While the paper does not define separate named variants of cohesion or coupling, several practical variants are implied through the design tasks \cite{Li2024FullSDLC}:

\textbf{1. Cohesion Variants}
\begin{itemize}
    \item \textit{Class-level cohesion:} Evaluates whether the responsibilities within a class logically belong together.
    \item \textit{Sequence-level cohesion:} Evaluates whether interactions in a sequence diagram are logically coherent.
    \item \textit{Architectural cohesion:} Evaluates whether files and modules are grouped in a way that reflects functional boundaries.
\end{itemize}

\textbf{2. Decoupling (Coupling) Variants}
\begin{itemize}
    \item \textit{Class-level decoupling:} Whether classes avoid unnecessary dependencies.
    \item \textit{Subsystem / module decoupling:} Evaluates dependency boundaries across significant components.
    \item \textit{File tree decoupling:} Assesses separation of concerns in the architecture design.
\end{itemize}

In each case, the principles are applied qualitatively by the LLM judge during pairwise comparison.

\subsubsection{Applications in Software Engineering}

Cohesion and decoupling play central roles across several software engineering contexts:

\textbf{1. Early Design Quality Assessment} \\
DevEval applies these principles directly to the \textit{Software Design} task. The benchmark emphasizes that high-quality designs must:
\begin{itemize}
    \item group related behaviors together,
    \item avoid unnecessary cross-class dependencies,
    \item maintain modular and readable structures.
\end{itemize}
These principles align with longstanding SE best practices such as the Single Responsibility Principle (SRP) and modular system design \cite{Li2024FullSDLC}.

\textbf{2. Long-Context Development} \\
Because LLMs in DevEval must maintain design consistency across multiple diagrams and artifacts, cohesion and decoupling serve as indicators of:
\begin{itemize}
    \item whether the model can maintain structural reasoning over large contexts,
    \item whether design decisions propagate consistently.
\end{itemize}

\textbf{3. Impact on Implementation and Testing} \\
Highly cohesive and decoupled designs lead to:
\begin{itemize}
    \item clearer implementations,
    \item fewer cross-file errors,
    \item more maintainable and testable code.
\end{itemize}
Thus, cohesion and decoupling indirectly support later DevEval tasks (implementation, acceptance testing, unit testing) \cite{Li2024FullSDLC}.

\subsubsection{Interpretation}

\textbf{What a high cohesion score means}
\begin{itemize}
    \item Classes and modules each have a focused, unified purpose.
    \item Responsibilities are not scattered.
    \item Sequence diagrams follow coherent interactions.
    \item Architectural layouts reflect logical groupings.
\end{itemize}

\textbf{What a high decoupling score means}
\begin{itemize}
    \item Classes and modules are independent.
    \item Few unnecessary dependencies.
    \item Clear separation of concerns.
    \item Architecture avoids entanglement.
\end{itemize}

\textbf{Advantages}
\begin{itemize}
    \item Reflects human-understandable design quality \cite{Li2024FullSDLC}.
    \item Connects closely with classic SE metrics.
    \item Captures structural reasoning that pure code metrics cannot.
    \item Effective at evaluating long-context LLM capabilities.
\end{itemize}

\textbf{Limitations}
\begin{itemize}
    \item DevEval uses qualitative LLM-as-a-judge scoring, not numerical formulas \cite{Li2024FullSDLC}.
    \item Evaluating cohesion and coupling can involve subjective judgment.
    \item No automated parser-based metrics are provided in the paper.
    \item Quality may depend on prompt wording and model biases in the judge.
\end{itemize}

Even with these limitations, cohesion and decoupling provide an essential view of design quality that complements implementation-based evaluation.