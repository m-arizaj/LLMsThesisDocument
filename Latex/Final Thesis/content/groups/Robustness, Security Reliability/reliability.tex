\subsection{Reliability}

Reliability is one of the core evaluation dimensions discussed in \textit{Overview of the Comprehensive Evaluation of Large Language Models}. The paper defines reliability as a multi-component quality that determines whether a model can be trusted to produce stable, accurate, and faithful outputs under repeated or slightly varied conditions \cite{Lin2024SWC}.

The authors describe Reliability as the integration of five critical sub-qualities:

\begin{quote}
“Reliability… including accuracy, hallucination, consistency, interpretability, and calibration error.” \cite{Lin2024SWC}
\end{quote}

This definition positions Reliability as a composite behavioral property, capturing both correctness and stability of a model's outputs. It evaluates whether the model behaves predictably, avoids fabrications, maintains coherence, and expresses its confidence appropriately.

The paper does not define a formal mathematical formula or quantitative scoring method for Reliability. It is presented as a \textit{conceptual evaluation dimension}, composed of five qualitative components \cite{Lin2024SWC}:
\begin{itemize}
    \item \textit{Accuracy}
    \item \textit{Hallucination}
    \item \textit{Consistency}
    \item \textit{Interpretability}
    \item \textit{Calibration Error}
\end{itemize}

Thus, Reliability can be expressed conceptually as:

\begin{equation}
\text{Reliability} = f(\text{Accuracy},\;
\text{Hallucination},\;
\text{Consistency},\;
\text{Interpretability},\;
\text{CalibrationError})
\end{equation}

However, no operationalization or normalization procedure is provided in the paper. There is no weighting scheme, no aggregation function, and no dataset-specific evaluation rubric. Reliability functions as a high-level diagnostic dimension, not a precise metric.

\subsubsection{Variants}

While the paper does not define separate variants of Reliability, its five components act as \textit{internal sub-metrics} with distinct conceptual roles \cite{Lin2024SWC}:

\textbf{1. Accuracy} \\
Measures factual or task-specific correctness.

\textbf{2. Hallucination} \\
Represents fabricated or unsupported output. Lower hallucination improves Reliability.

\textbf{3. Consistency} \\
Defined explicitly as:
\begin{quote}
“Consistency measures the stability and predictability of a model’s output under the same or similar conditions.” \cite{Lin2024SWC}
\end{quote}
Consistency ensures the model produces repeatable responses across identical prompts.

\textbf{4. Interpretability} \\
Relates to how understandable or explainable the model’s reasoning or response is.

\textbf{5. Calibration Error} \\
Concerns whether the model’s confidence reflects actual correctness.

These components do not constitute variants but rather \textit{dimensions inside the Reliability umbrella}.

\subsubsection{Applications in Software Engineering}

Reliability is critically important in software engineering contexts because:

\textbf{1. Reproducibility of results} \\
Developers expect stable outputs across repeated prompts. High consistency prevents “prompt drift” in iterative workflows.

\textbf{2. Safety and correctness} \\
Reducing hallucination is necessary to avoid generating invalid APIs, unsafe code patterns, or nonexistent library functions.

\textbf{3. Trust and debugging} \\
Predictable model behavior improves trust during:
\begin{itemize}
    \item unit test generation,
    \item code explanation,
    \item refactoring,
    \item bug detection.
\end{itemize}

\textbf{4. Confidence alignment} \\
Proper calibration helps developers interpret the model's likelihood of correctness, crucial when models propose:
\begin{itemize}
    \item complex patches,
    \item architectural designs,
    \item migration strategies.
\end{itemize}

Although the paper does not provide software-specific examples, Reliability maps cleanly onto SE tasks where stability and correctness are non-negotiable.

\subsubsection{Interpretation}

\textbf{1. High Reliability indicates:}
\begin{itemize}
    \item Stable outputs across identical prompts (high consistency).
    \item Accurate responses with low factual and code hallucination.
    \item Explanations that users can understand (interpretability).
    \item Confidence expressions that match correctness likelihood (calibration).
    \item Predictable behavior across contexts.
\end{itemize}

\textbf{2. Low Reliability indicates:}
\begin{itemize}
    \item Different answers to the same query (low consistency).
    \item Fabricated facts or invalid code suggestions (hallucination).
    \item Confusing or opaque reasoning (low interpretability).
    \item Overconfident or underconfident predictions (calibration error).
\end{itemize}

Because Reliability aggregates multiple sub-qualities, it reflects the model’s overall trustworthiness \cite{Lin2024SWC}.