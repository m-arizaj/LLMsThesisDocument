\subsection{Performance (Functional \& Efficiency)}

\subsubsection*{Introduction}
In the evaluation of Large Language Models, particularly in code generation and software engineering domains, \textbf{Performance} is a broad category of metrics. It is used to quantify the model's success in two key areas:

\begin{itemize}
    \item \textbf{Functional Performance / Correctness}: Does the code generated by the model work as intended? This is typically measured by executing the code against a set of test cases.
    \item \textbf{Efficiency Evaluation}: How efficient is the model in its code generation process, especially when measured against a baseline?
\end{itemize}

This category includes specific metrics based on test pass rates and differential scores that compare a model against a standard.

\subsubsection*{1. Functional Correctness Metrics}
These metrics assess whether the model's output behaves according to requirements.

\textbf{Functional Performance (General)}
\begin{itemize}
    \item \textit{Definition}: A general metric used to assess whether the model's output (e.g., a function, class, or patch) behaves according to requirements, often validated by execution.
    \item \textit{Applications}: Used in benchmarks like \textbf{DevQualityEval}.
\end{itemize}

\textbf{Unit Test Performance}
\begin{itemize}
    \item \textit{Definition}: A specific measure that evaluates whether the code passes a predefined set of fine-grained tests checking individual components (e.g., a single function) in isolation.
    \item \textit{Applications}: Used in benchmarks like \textbf{LoCoBench} for long-context tasks.
\end{itemize}

\textbf{Integration Test Performance}
\begin{itemize}
    \item \textit{Definition}: Unlike unit tests, this evaluates the model's ability to generate code that functions correctly when combined with other parts of a larger system. It checks the interactions between different modules.
    \item \textit{Applications}: Used in benchmarks like \textbf{LoCoBench} to assess broader software context integration.
\end{itemize}

\subsubsection*{2. Efficiency Evaluation Metrics}
These metrics measure the resource usage or speed of the generated code compared to a baseline.

\textbf{Differential Performance Score}
\begin{itemize}
    \item \textit{Definition}: Designed to measure the \textit{difference} in performance (such as execution speed or resource usage) between the code generated by an LLM and a baseline or reference implementation.
    \item \textit{Applications}: Used in the \textbf{EVALPERF} benchmark.
\end{itemize}

\textbf{Normalized Differential Performance Score}
\begin{itemize}
    \item \textit{Definition}: A variant of the differential score that normalizes the value. This allows for a standardized comparison of performance differences across different tasks or models that might have different scales.
    \item \textit{Applications}: Also used within the \textbf{EVALPERF} benchmark.
\end{itemize}

\subsubsection*{Comparative Summary}
Below is the classification of performance metrics by category and benchmark:

\begin{center}
\begin{tabular}{|p{3cm}|p{2cm}|p{2.5cm}|p{5cm}|}
\hline
\textbf{Metric} & \textbf{Category} & \textbf{Benchmark} & \textbf{Purpose} \\
\hline
Functional Performance & Functional Correctness & DevQualityEval & Assesses if generated code works as intended. \\
\hline
Unit Test Performance & Functional Correctness & LoCoBench & Assesses correctness at the individual function/method level. \\
\hline
Integration Test Perf. & Functional Correctness & LoCoBench & Assesses correctness of interactions between code components. \\
\hline
Diff. Performance Score & Efficiency Evaluation & EVALPERF & Measures the performance \textit{difference} against a baseline. \\
\hline
Norm. Diff. Perf. Score & Efficiency Evaluation & EVALPERF & Provides a \textit{standardized} score for performance differences. \\
\hline
\end{tabular}
\end{center}


\subsubsection{Additional References}

This metric is referenced and/or used in the following paper(s):


\sloppy
\cite{
Bistarelli2025UsageLLMCode,
Liu2024EfficientCodeGeneration,
Qiu2025LoCoBench,
}
\fussy
