\subsection{Security Analysis Score (SAS)}

The \textbf{Security Analysis Score (SAS)} is a metric used to evaluate the security posture of code, typically generated by a Large Language Model (LLM). It provides a quantitative vulnerability assessment based on established industry frameworks, most notably the \textbf{OWASP Top 10} (Open Web Application Security Project).

The score specifically evaluates the presence or absence of common, critical security issues, such as:
\begin{itemize}
    \item SQL Injection
    \item Cross-Site Scripting (XSS)
    \item Buffer Overflows
    \item Insecure Cryptographic Practices
\end{itemize}

Unlike metrics with a simple arithmetic formula, the SAS is derived from the output of \textbf{Static Analysis Security Testing (SAST)} tools.

The calculation is generally based on the \textbf{identification, count, and severity classification} of vulnerabilities found in the code.
\begin{itemize}
    \item \textbf{Higher Score}: Indicates fewer or less severe vulnerabilities (better security).
    \item \textbf{Lower Score}: Indicates significant security risks.
\end{itemize}

It acts as a key component of the ``Code Quality Assessment'' dimension in modern software engineering benchmarks like LoCoBench.

\subsubsection*{Purpose}
The primary purpose of the SAS is to move evaluation beyond mere functional correctness and assess the \textbf{safety and robustness} of generated code. It quantifies a model's ability to produce code that is not only functional but also secure against common attack vectors, a critical aspect of real-world software engineering.

\subsubsection*{Domains}
\begin{itemize}
    \item Software Engineering
    \item Long-Context Code Evaluation
    \item Code Quality Assessment
    \item Security Vulnerability Assessment
\end{itemize}

\subsubsection{Applications}

In the context of modern software engineering and LLM benchmarking, the SAS is applied in several critical workflows:

\begin{itemize}
    \item \textbf{Automated Quality Assurance Pipelines}: SAS is utilized as a gatekeeping metric in automated validation pipelines (such as Phase 4 of the LoCoBench pipeline). It integrates with language-specific static analysis tools (e.g., Bandit for Python, SpotBugs for Java) to automatically flag and filter generated scenarios that contain security flaws, ensuring that synthetic codebases meet minimum safety standards without human intervention \cite{Qiu2025LoCoBench}.

    \item \textbf{Benchmarking Model Safety}: It serves as a comparative metric to rank LLMs based on their ability to adhere to security best practices. By quantifying the frequency of vulnerabilities like XSS or SQL injection in generated code, SAS allows engineers to identify which models prioritize defensive programming and which are prone to generating insecure patterns \cite{Qiu2025LoCoBench}.

    \item \textbf{Long-Context Vulnerability Assessment}: SAS is applied to evaluate how well models maintain security constraints over extended contexts (up to 1M tokens). It helps determine if a model's ability to track security requirements degrades as the complexity and size of the codebase increase, which is crucial for enterprise-scale software development \cite{Qiu2025LoCoBench}.

    \item \textbf{Threat Vector Identification Tasks}: Within specific ``Security Analysis'' task categories, SAS is used to measure a model's proficiency in identifying vulnerabilities within existing code and implementing appropriate mitigation strategies, effectively simulating the role of a security auditor \cite{Qiu2025LoCoBench}.
\end{itemize}

\subsubsection*{Advantages}
\begin{itemize}
    \item \textbf{Real-World Relevance}: Measures a critical aspect of code quality (security) often overlooked by metrics focused only on correctness or similarity.
    \item \textbf{Standardized}: Basing the assessment on frameworks like OWASP provides a well-understood and widely accepted basis for evaluation.
    \item \textbf{Actionable}: Provides specific feedback on security flaws (e.g., ``SQL injection found''), enabling targeted model improvement.
\end{itemize}

\subsubsection*{Limitations}
\begin{itemize}
    \item \textbf{Tool-Dependent}: The score's accuracy is highly dependent on the underlying static analysis tools used.
    \item \textbf{False Positives/Negatives}: Static analysis may misidentify issues or miss complex, logic-based vulnerabilities that require dynamic execution.
    \item \textbf{Scope}: Primarily focused on common vulnerabilities (OWASP Top 10) and may not cover novel or highly domain-specific flaws.
\end{itemize}


\subsubsection{Additional References}

This metric is referenced and/or used in the following paper(s):


\sloppy
\cite{
Qiu2025LoCoBench,
}
\fussy

