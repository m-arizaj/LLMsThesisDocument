\subsection{Factuality}

\subsubsection*{Definition}
\textbf{Factuality} in the context of Large Language Models (LLMs) refers to the extent to which the information or answers provided by the model align with \textbf{real-world truths and verifiable facts}.

Evaluating factuality is a critical component of LLM assessment. It measures the model's ability to:
\begin{itemize}
    \item Maintain consistency with known facts.
    \item Avoid generating misleading or false information, a phenomenon known as ``factual hallucination.''
    \item Effectively learn and recall factual knowledge.
\end{itemize}

\subsubsection*{Purpose}
The primary purpose of evaluating factuality is to \textbf{ensure trust} and enable the efficient use of LLMs. Factuality significantly impacts the reliability of downstream applications. Inconsistent or incorrect information can lead to substantial misunderstandings, making this metric crucial.

\subsubsection*{Domains \& Applications}
This metric is applied across general LLM evaluation and specific high-stakes domains:
\begin{itemize}
    \item \textbf{Question Answering (QA)}: Ensuring answers are grounded in reality.
    \item \textbf{Text Summarization}: Ensuring the summary remains factually consistent with the source.
    \item \textbf{Automated Fact-Checking}: Using models to verify claims.
    \item \textbf{Information Extraction}: Retrieving accurate data points.
\end{itemize}

\subsubsection*{Benchmarks}
Several key benchmarks are used to assess factuality:
\begin{itemize}
    \item \textbf{HELM (Holistic Evaluation of Language Models)}: Evaluating general capabilities including accuracy.
    \item \textbf{TruthfulQA}: A dataset specifically designed to cause models to make mistakes and test truthfulness (mimicking human falsehoods).
    \item \textbf{Natural Questions} \& \textbf{TriviaQA}: Used to assess internal knowledge capabilities.
    \item \textbf{FActScore}: A metric/benchmark that breaks down generated text into individual ``atomic'' facts which are then evaluated for correctness.
\end{itemize}

\subsubsection*{Advantages}
\begin{itemize}
    \item \textbf{Builds Trust}: Directly measures the reliability and truthfulness of a model, which is essential for user trust.
    \item \textbf{Critical for Applications}: Serves as a gate for deploying LLMs in information-sensitive fields like medicine, finance, and education.
    \item \textbf{Targets Hallucination}: Provides a direct way to quantify and track a model's tendency to ``hallucinate'' or fabricate information.
\end{itemize}

\subsubsection*{Limitations}
\begin{itemize}
    \item \textbf{No Unified Framework}: There is an absence of a unified comparison framework for factual consistency.
    \item \textbf{Scaling Isn't a Silver Bullet}: Simply scaling up model sizes does not necessarily improve their truthfulness.
    \item \textbf{Estimator Performance}: Current estimators designed to measure factuality (like FActScore) still have limitations in effectively addressing the task.
    \item \textbf{Hallucination Risk}: Models are capable of generating coherent-sounding text that includes factual inaccuracies, making detection difficult.
\end{itemize}

% añadir estas entradas a tu archivo .bib

% @article{lin2021truthfulqa,
%   title={TruthfulQA: Measuring how models mimic human falsehoods},
%   author={Lin, S. and Hilton, J. and Evans, O.},
%   journal={arXiv preprint arXiv:2109.07958},
%   year={2021},
%   doi={10.48550/arXiv.2109.07958}
% }

% @article{wang2023evaluating,
%   title={Evaluating open question answering evaluation},
%   author={Wang, C. and Cheng, S. and Xu, Z. and others},
%   journal={arXiv preprint arXiv:2305.12421},
%   year={2023},
%   doi={10.48550/arXiv.2305.12421}
% }

% @article{min2023factscore,
%   title={FActScore: Fine-grained atomic evaluation of factual precision in long form text generation},
%   author={Min, S. and Krishna, K. and Lyu, X. and others},
%   journal={arXiv preprint arXiv:2305.14251},
%   year={2023},
%   doi={10.48550/arXiv.2305.14251}
% }

% @article{honovich2022true,
%   title={TRUE: Re-evaluating factual consistency evaluation},
%   author={Honovich, O. and Aharoni, R. and Herzig, J. and others},
%   journal={arXiv preprint arXiv:2204.04991},
%   year={2022},
%   doi={10.48550/arXiv.2204.04991}
% }

% @article{gekhman2023trueteacher,
%   title={TrueTeacher: Learning factual consistency evaluation with large language models},
%   author={Gekhman, Z. and Herzig, J. and Aharoni, R. and others},
%   journal={arXiv preprint arXiv:2305.11171},
%   year={2023},
%   doi={10.48550/arXiv.2305.11171}
% }

% @article{manakul2023selfcheckgpt,
%   title={SelfCheckGPT: Zero-resource black-box hallucination detection for generative large language models},
%   author={Manakul, P. and Liusie, A. and Gales, M.},
%   journal={arXiv preprint arXiv:2303.08896},
%   year={2023},
%   doi={10.48550/arXiv.2303.08896}
% }