\subsection{Difference Automata}

\subsubsection*{Introduction}
Difference Automata are a validation metric used in software engineering, particularly for verifying system migration and ensuring behavioral equivalence. Unlike metrics that produce a single score, a Difference Automaton is a complete, learnable behavioral model (specifically a Mealy or Moore machine).

This metric is generated by simultaneously testing two versions of a system (e.g., a source and a target) to ``fully characterize the behavioral discrepancies between the two compared models.'' The approach is utilized by Busch et al. (2025) to automatically validate the correctness of an LLM-based migration from JavaScript to TypeScript.

A key derivative of this model is the \textbf{Nonempty Difference Automaton Indicator}, which serves as a direct and intuitive signal for detecting migration errors.

\subsubsection*{Definition}
The concept encompasses two related metric definitions:

\begin{itemize}
    \item \textbf{Difference Automata}: A behavioral model (e.g., a Mealy or Moore machine) inferred by ``simultaneously testing two systems for discrepancies.'' It is designed to fully characterize the behavioral differences between an original and a migrated application, showing all traces that lead to divergent behavior.
    \item \textbf{Nonempty Difference Automaton Indicator}: A derived binary metric based on the Difference Automata model. It is a simple validation check to determine if the inferred difference automaton is ``nonempty'' (i.e., contains any states or transitions indicating a difference).
\end{itemize}

\subsubsection*{Purpose}
\begin{itemize}
    \item \textbf{Behavioral Equivalence Verification}: The Difference Automaton is ideal for quality control because it clearly visualizes behavioral differences before and after migration. It detects discrepancies resulting from the LLM-supported migration process.
    \item \textbf{Migration Error Detection}: The Nonempty Indicator serves a binary purpose: ``Any nonempty difference automaton serves as an intuitive indicator of an erroneous migration.'' Conversely, a migration is deemed correct if no input yields a nonempty difference automaton.
\end{itemize}

\subsubsection*{Applications}
\begin{itemize}
    \item \textbf{System Migration Validation}: Specifically used for validating the LLM-based migration of Webstory applications from JavaScript to TypeScript.
    \item \textbf{Model Comparison}: Systematically comparing models generated before and after migration to isolate faults.
    \item \textbf{Signaling}: The Nonempty Indicator acts as the primary signal or ``most convenient way'' to alert developers to behavioral discrepancies introduced during migration.
\end{itemize}

\subsubsection*{Limitations}
\begin{itemize}
    \item The validation method relies on the scalability of \textit{Active Automata Learning} (AAL). This requires managing the size of the learning alphabets and the inferred models to remain computationally feasible.
\end{itemize}

\subsubsection*{Comparative Summary}
Below is a comparison of the Difference Automata and its derived indicator:

\begin{center}
\begin{tabular}{|p{2.5cm}|p{2.2cm}|p{3cm}|p{3.5cm}|p{2.5cm}|}
\hline
\textbf{Metric} & \textbf{Based on} & \textbf{Goal} & \textbf{Output} & \textbf{Typical Domain} \\
\hline
Difference Automata & Active Automata Learning & Behavioral Equivalence Verification & A behavioral model showing all divergent traces & Software Migration / Validation \\
\hline
Nonempty Indicator & Difference Automata & Migration Error Detection & A binary signal indicating if an error was found & Software Migration / Validation \\
\hline
\end{tabular}
\end{center}

\subsubsection{Additional References}

This metric is referenced and/or used in the following papers:

\sloppy
\cite{
Busch2025LLMCodeMigration,
}
\fussy