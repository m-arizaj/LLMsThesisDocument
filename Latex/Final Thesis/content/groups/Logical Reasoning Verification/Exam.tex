\subsection{EXAM (Expected Maximum Fault Localization)}

Based on the survey by Chen et al., EXAM stands for \textbf{Expected Maximum Fault Localization}. It is a evaluation metric used in the field of software engineering, specifically for the task of \textbf{Fault Localization (FL)}.

In contrast to benchmarks or datasets, EXAM is a scoring mechanism used to assess the effectiveness of deep learning-based fault localization (DLFL) techniques.

\subsubsection*{Purpose}
The purpose of EXAM is to quantitatively evaluate how effective a model is at locating bugs in source code. Specifically, it measures the ``expected rank of the first correct fault location in a ranked list of code elements''. This helps researchers understand the effort required by a developer to find the first bug when inspecting the code elements ranked by the model.

\subsubsection*{Metrics Used within EXAM Context}
In the context of Fault Localization, EXAM is often used alongside other ranking metrics to make a score a model's performance. The provided text highlights the following frequently used metrics:

\begin{itemize}
    \item \textbf{EXAM}: Measures the expected rank of the first correct fault location in a ranked list of code elements.
    \item \textbf{Top-N}: Measures the number of times the model identifies the correct faulty code element within the top-N ranked elements.
    \item \textbf{MAR (Mean Average Rank)}: Measures the mean of the average rank of the faults.
    \item \textbf{MFR (Mean First Rank)}: Measures the mean of the first faulty statement's rank for all faults using a localization approach.
\end{itemize}

\subsubsection{Applications}

Based on the provided survey, the EXAM metric is applied in the following area of software engineering:

\begin{itemize}
    \item \textbf{Fault Localization (FL)}: EXAM metric is extensively used to evaluate the effectivity of automated fault localization techniques. In this context, it serves as a proxy for ``developer effort'' by estimatin the percentage of code a developer would need to inspect to find the first bug in a ranked list of suspicious code elements \cite{Chen2024DLBasedSE}.
\end{itemize}

\sloppy
\cite{
Chen2024DLBasedSE,
}
\fussy